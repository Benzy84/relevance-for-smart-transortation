\documentclass{article}
\usepackage[utf8]{inputenc}
\usepackage{wrapfig}
\usepackage{graphicx}
\usepackage{setspace}
\usepackage{lipsum}
\usepackage[skip=0.333\baselineskip]{caption}
\usepackage{subcaption}
\usepackage[colorlinks=true, linkcolor=blue, citecolor=blue, urlcolor=blue]{hyperref}
\usepackage[style=numeric,backend=bibtex,giveninits=true,terseinits=true,doi=false,url=false,isbn=false]{biblatex}
\usepackage{geometry}
\geometry{top=1in, bottom=1in, left=0.75in, right=0.75in}

\title{\textbf{\Large Relevance of Matrix-based Analysis in LiDAR to Smart Transportation and Zero Transportation Externalities Vision}}
\author{\textbf{Benzy Laufer} \\ Supervisor - Prof. Ori Katz \\ Institute of Applied Physics, The Hebrew University of Jerusalem.}

\begin{document}
\maketitle

\section{Introduction}
The proposed research aims to improve the Signal-to-Noise Ratio (SNR) and resolution in Light Detection and Ranging (LiDAR) technology through matrix-based analysis. This advancement is particularly relevant to the field of smart transportation and aligns with the vision of Zero Transportation Externalities.

\section{Relevance to Smart Transportation}
LiDAR technology is integral to the development of autonomous vehicles and smart transportation systems. By enhancing the performance of LiDAR through matrix-based analysis, we can significantly improve the mapping and detection capabilities of these systems. This leads to better obstacle detection and avoidance, improved route planning, and more efficient traffic management. 

\section{Advancement of Zero Transportation Externalities Vision}
The improvement in LiDAR technology contributes to the vision of Zero Transportation Externalities in three ways:

\begin{enumerate}
    \item \textbf{Zero casualties:} Improved obstacle detection and avoidance can lead to safer transportation, reducing the number of accidents and casualties.
    \item \textbf{Zero delays:} Enhanced mapping and navigation capabilities can minimize traffic congestion and delays, ensuring more efficient transportation.
    \item \textbf{Zero environmental harm:} By facilitating the development and adoption of autonomous vehicles, the research can help optimize fuel consumption, reduce emissions, and promote the use of electric vehicles, contributing to a greener transportation system.
\end{enumerate}

\section{Matrix-based Analysis in LiDAR}
The research focuses on the adaptation and development of advanced matrix-based techniques for LiDAR imaging. These techniques, including Image Scanning Microscopy (ISM) and aberration estimation and correction, have significantly improved resolution, SNR, and image quality in various fields. Despite their success, these techniques have yet to be applied to LiDAR due to its requirement for multi-pixel detection. However, recent advancements in multi-pixel detector technology now make it possible to implement these techniques in the field of LiDAR.

LiDAR, operating in the 'photon starved regime', can greatly benefit from these matrix-based techniques. By using multi-pixel detection, we are able to utilize the entire detector array without reducing the resolution, allowing us to collect more photons than conventional LiDAR. Furthermore, the matrix-based approach allows us to handle aberrations and scatterings in real-world scenarios, thereby enhancing the robustness and reliability of the LiDAR system in smart transportation applications.





\end{document}
